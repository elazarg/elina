\documentclass[twoside,10pt,a4paper]{report}
\usepackage[latin1]{inputenc}
\usepackage[T1]{fontenc}
\usepackage{ae}
\usepackage{fullpage}
\usepackage{url}
\usepackage{ocamldoc}
\usepackage{makeidx}

\usepackage{fancyhdr}
\pagestyle{fancy}
\renewcommand{\headrulewidth}{0.9pt}
\renewcommand{\footrulewidth}{0pt}
\setlength{\headheight}{2.8ex}
\setlength{\footskip}{5ex}
\renewcommand{\chaptermark}[1]{ %
  \markboth{\MakeUppercase{\chaptername}\ \thechapter.\ #1}{}}
\renewcommand{\sectionmark}[1]{}
\setcounter{tocdepth}{1}
\setcounter{secnumdepth}{4}
\usepackage{color}
\definecolor{mygreen}{rgb}{0,0.6,0}

\usepackage[ps2pdf]{hyperref}

\setlength{\parindent}{0em}
\setlength{\parskip}{0.5ex}

%\usepackage{listings}
%\lstloadlanguages{Caml}

\makeindex

\title{MLApronIDL: OCaml interface for APRON library}
\author{Bertrand Jeannet}

\begin{document}
\maketitle

\vspace*{0.9\textheight}

All files distributed in the APRON library, including \textsc{MLApronIDL}
subpackage, are distributed under LGPL license.

Copyright (C) Bertrand Jeannet 2005-2006 for the
\textsc{MLApronIDL} subpackage.

\newpage

\tableofcontents

%*****************************************************************************
\chapter{Introduction}
%*****************************************************************************

This package is an \textsc{OCaml} interface for the APRON
library/interface.  The interface is accessed via the module
Apron, which is decomposed into 15 submodules, corresponding to C
modules:

\noindent
\begin{tabular}{l@{~:~~}l}
Scalar     & scalars (numbers) \\
Interval   & intervals on scalars \\
Coeff      & coefficients (either scalars or intervals) \\
Dimension  & dimensions and related operations \\
Linexpr0   & (interval) linear expressions, level 0 \\
Lincons0   & (interval) linear constraints, level 0 \\
Generator0 & generators, level 0 \\
Manager    & managers \\
Abstract0  & abstract values, level 0 \\
Var        & variables \\
Environment& environment binding variables to dimensions \\
Linexpr1   & (interval) linear expressions, level 1 \\
Lincons1   & interval) linear constraints, level 1 \\
Generator1 & generators, level 1 \\
Abstract1  & abstract values, level 1 \\
Parser     & strings parsing 
\end{tabular}

%=============================================================================
\subsection*{Requirements}
%=============================================================================

\begin{itemize}
\item M4 preprocessor (standard on any UNIX system)
\item APRON library
\item GMP library (tested with version 4.0 and up)
\item mlgmpidl package
\item OCaml 3.0 or up (tested with 3.09)
\item Camlidl (tested with 1.05)
\end{itemize}

%=============================================================================
\subsection*{Installation}
%=============================================================================

\begin{description}
\item[Library]
Set the file ../Makefile.config to your own setting.
You might also have to modify the Makefile for executables

type 'make', and then 'make install'

The OCaml part of the library is named apron.cma (.cmxa, .a) The C
part of the library, which is automatically referenced by
apron.cma/apron.cmxa, is named libapron\_caml.a
(libapron\_caml\_debug.a)

'make install' installs not only .mli, .cmi, but also .idl files.
\item[Documentation]
The documentation (currently very sketchy) is generated with ocamldoc.

'make mlapronidl.pdf'

'make html' (put the HTML files in the html subdirectoy)

\item[Miscellaneous]
'make clean' and 'make distclean' have the usual behaviour.
\end{description}

%=============================================================================
\subsection*{Compilation and Linking}
%=============================================================================

To make things clearer, we assume an example file
\texttt{example.ml} which uses both \textsc{NewPolka} (convex
polyhedra) and \textsc{Box} (intervals) libraries, in their
versions where integers (resp. rationals) are GMP integers (resp.
rationals). We assume that C and OCaml interface and library files
are located in directory \$APRON/lib.

The native-code compilation command looks like
\begin{quote}\tt
  ocamlopt -I \$APRON/lib -o exampleg.opt bigarray.cmxa gmp.cmxa apron.cmxa box.cmxa polka.cmxa -cclib "-lpolkag -lboxmpq"
\end{quote}
Comments:
\begin{enumerate}
\item You need at least the libraries \texttt{bigarray} (standard
  \textsc{OCaml} distribution), \texttt{gmp}, and \texttt{apron}
  (standard APRON distribution), plus the one implementing an
  effective abstract domains: here, \texttt{box}, and
  \texttt{polka}.
\item The C libraries associated to those \textsc{OCaml} libraries
  (\textit{e.g.,} \texttt{gmp\_caml}, \texttt{box\_caml}, \ldots)
  are automatically looked for, with the exception of the
  libraries implementing abstract domains (\textit{e.g.,}
  \texttt{polkag}, \texttt{boxmpq}). The reason is that most of
  these libraries may be compiled with different internal
  representations and/or options, while their interface remains
  the same.
\end{enumerate}

The byte-code compilation process looks like
\begin{quote}\tt
  ocamlc -I \$APRON/lib -make-runtime -o myrun bigarray.cma gmp.cma apron.cma box.cma polka.cma -cclib "-lpolkag -lboxmpq" \\
  ~\\
  ocamlc -I \$APRON/lib -use-runtime myrun -o exampleg bigarray.cma gmp.cma apron.cma box.cma polka.cma -cclib "-lpolkag -lboxmpq" example.ml
\end{quote}
Comments:
\begin{enumerate}
\item One first build a custom bytecode interpreter that includes the new native-code needed;
\item One then compile the \texttt{example.ml} file.
\end{enumerate}

The automatic search for C libraries associated to these
\textsc{OCaml} libraries can be disabled by the option
\texttt{-noautolink} supported by both \texttt{ocamlc} and
\texttt{ocamlopt} commands. For instance, the command for native-code compilation can alternatively looks like:
\begin{quote}\tt
  ocamlopt -I \$APRON/lib -noautolink -o exampleg.opt bigarray.cmxa gmp.cmxa apron.cmxa box.cmxa polka.cmxa -cclib "-lpolkag -polka\_caml -lboxmpq -lbox\_caml -lapron\_caml -lapron -lgmp\_caml -LGMP/lib -lgmp"
\end{quote}

The option \texttt{-verbose} helps to understand what is happening
in case of problem.

\part{Coefficients}
\input{Mpzf.tex}
\input{Mpqf.tex}
\input{Scalar.tex}
\input{Interval.tex}
\input{Coeff.tex}

\part{Managers and Abstract Domains}
\input{Manager.tex}
\input{Box.tex}
\input{Oct.tex}
\input{Polka.tex}

\part{Level 1 of the interface}
\input{Var.tex}
\input{Environment.tex}
\input{Linexpr1.tex}
\input{Lincons1.tex}
\input{Generator1.tex}
\input{Abstract1.tex}
\input{Parser.tex}

\part{Level 0 of the interface}
\input{Dim.tex}
\input{Linexpr0.tex}
\input{Lincons0.tex}
\input{Generator0.tex}
\input{Abstract0.tex}

\part{Others}
\input{Mpz.tex}
\input{Mpq.tex}
\input{Gmp_random.tex}

\appendix
\printindex
\end{document}
